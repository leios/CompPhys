\chapter{A note before we start}

When I was younger, I wanted to be an author. 
I would write for hours and hours every day, becoming incrementally better all the time. 
Man, those were some fun times! 
To this day, I can think of nothing more fun than picking up a pen and scribbling down a good story; however, I am no author.
Instead, I am a physicist... Well, maybe I'm a programmer. 

I suppose I am what you might call a computational physicist, an individual who studies physics through programming / simulating it.
I've got to admit, it's a lot of fun. 
Seriously, I love pouring over old texts or new papers, trying to understand obscure algorithms that no one ever cared to explain in a clear or understandable way. 
Maybe that's why I'm writing this book to begin with -- to make the process of learning physics algorithms just a little easier for students in the future.

But there's more to it than that. 
I am a person who enjoys learning, and throughout my life, I have always enjoyed trying to become better at something.
I honestly believe that most people enjoy the process of learning, we have just been taught to dislike it.
This might be why so many people find themselves engrossed in some fantasy or video game universe -- those stories are simulations of our own universe with subtle changes to make them more understandable and interesting.
We love learning about these universes, so why not our own?

Maybe because our own universe is too complicated and we become overloaded with information. 
Many times throughout my life, I can remember becoming incredibly frustrated at my own inability to understand nature.
Everywhere I turned, I had questions. 
Each one of these questions led me to a labyrinth of twisted passages, eventually leading to a solid, impassable wall.
Whenever I asked for help, I was not gently guided to end.
Instead I was told, ``You are wrong" and forced onto the solution without understanding how or why I got there to begin with. 
All said, it was an incredibly unmotivating method of learning.
Maybe that's why I liked writing in the first place. 
I could create a world where I knew all the rules. 
It was easy to get to the story and start a new adventure.

But somewhere along the line, I started to see the world differently. 
The labyrinth I was desperately trying to get out of before was a story in it's own right.
Instead of a number of unanswerable questions, I saw a collection of shortstories waiting to be written.
A library filled with empty books, and on the spine of each book, a question.

Nowadays, I still simulate the world in small ways, but now I do so with code instead of stories. 
It's more concrete and in some cases may even allow us to push the boundary of human knowledge ever so slightly forward.

I guess my hope is to motivate you, the reader, to join me in my journey to understand nature, one simulation at a time.
If this sounds interesting to you, then please read further. 
It's my intention to make this book as approachable as possible