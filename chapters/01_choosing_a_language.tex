\chapter{Choosing a language}

For someone who has never compiled code before, programming might seem like a daunting task. 
For many veterans, the challenge seems to start with the fundamentals of computing, such as loops, variables, functions and classes; however, my first hurdle came even before that -- I had to choose a language to learn.
Now, for those who first learned programming in an official course taught in a university or high school setting, this question would never have come to mind. 
The students learn what the professor teaches, easy decision!
That said, I can remember spending weeks researching the differences between programming languages.
There were so many different sources claiming so many different reasons one language would be preferred over the other.
Truth be told, I ended up being inundated with useless information for no reason and kept putting off the learning programming even though I wanted to do something!

At the time, I thought the choice of a programming language would forever warp my ability to code in the future, and because I didn't know which problems I wanted to solve with programming at the time, I was terrified of making the wrong choice. 
More than that, every time I googled \textit{``Can language \textbf{X} be used for \textbf{Y}?"}, the answer would always be \textit{``Yes! So long as you do \textbf{Z}"}.
It seemed to me that somehow all the languages were equivalent, which terrified me even more!
Gosh, I was a mess!
So, I guess that's the point of this chapter: to list the positives and negatives of each language in a way that makes choosing one easy for the beginner programmer / physicist.

For the most part, we will be focussing on the big languages used by physicists for numerics:
\begin{itemize}
\item FORTRAN
\item C/C++
\item GPU/CUDA
\item Python
\item Julia
\end{itemize}

If your favorite language is not on this list, please feel free to add it through a pull request on Github! 
There are notable languages missing, such as Mathematica and Matlab.
I simply am not comfortable enough with these languages to support them at the current time.

\section*{FORTRAN}
\section*{C/C++}
\section*{GPU/CUDA}
\section*{Python}
\section*{Julia}

\section*{The next Hurdle}
After the language has been chosen, the next hurdle is writing code in that language. 
Of course, this means learning the most iconic program ever written: \textit{``Hello World!"}
However, at this point, your choice in language will drastically change the following chapter, so I hope there was sufficient information in the previous chapter for you to make a decision.
If you have still not decided on a language, I will arbitrarily choose C++ for you.

For any other choice, please recompile this document with 

\begin{lstlisting}
make language_of_chouce
\end{lstlisting}

For example, if your choice is Python, use

\begin{lstlisting}
make python
\end{lstlisting}

where the language will be all lowercase because I like lowercase letters.