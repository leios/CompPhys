At the time of writing this, I am a PhD student at the Okinawa Institute of Science and Technology, studying in the Quantum Systems Unit.
I do not claim to be a computational physics guru, but I am learning everything I can about the craft now.
As I learn more, I have strongly desired a concise and readable guide to learning many algorithms, and though texts like \textit{Numerical Recipes} are decent guides, they are often not explicitly written for a physicist in mind.
For these reasons, I have decided to chronicle my adventures in learning different computational physics algorithms, with the hope of improving both my own skills and helping anyone else who might be in a similar situation.

At the time of writing, I have no intention of ever monetizing this content. 
Instead, I would like for the content to be as open and approachable as possible!
Computational Physics is a living, breathing field with many new techniques being developed every day. 
In addition, every physicist has their own preferred languages and techniques.
Many stick to the old-school FORTRAN77, others the trusty C. 
Some venture into new territory with GPU computing or new languages like Julia.
In the end, no matter what language you are using, the algorithms remain similar.
Because of this, I would like for the book to be written in a modular way. 

If a physicist comes by wanting to learn numerical techniques, they should be able to build a variant of the book by choosing the chapters and languages that suit them. 
Likewise, if a computer scientist wants to learn the algorithms involved in physics, they should be able to bind together the chapters of the book related to their own needs. 
Basically, I would like this book to be as complete and thorough as possible. 
The user should know what they need and can take only that, if they desire. 

My current hope is to update this book every time I learn a new algorithm and slowly build the essentials from there.
If you know more than I do, please let me know! 
If you are particularly adventurous, I would love to see some pull requests that either fix my code samples or introduce key algorithms!

Anyway, that's about all I have for now. 
Thanks for looking at what I've got so far and let me know if there's anything that needs updating!
